%%%%%%%%%%%%%%%%%%%%%%%%%%%%%%%%%%%%%%%%%%%%%%%%%%%%%%%%%%%%%%%%%%%%%%
% How to use writeLaTeX: 
%
% You edit the source code here on the left, and the preview on the
% right shows you the result within a few seconds.
%
% Bookmark this page and share the URL with your co-authors. They can
% edit at the same time!
%
% You can upload figures, bibliographies, custom classes and
% styles using the files menu.
%
%%%%%%%%%%%%%%%%%%%%%%%%%%%%%%%%%%%%%%%%%%%%%%%%%%%%%%%%%%%%%%%%%%%%%%

\documentclass[12pt]{article}

\usepackage{sbc-template}

\usepackage{graphicx,url}

%\usepackage[brazil]{babel}   
\usepackage[utf8]{inputenc}  

     
\sloppy

\title{Finding Assortative Properties in Vehicular Networks}

\author{Felipe M. Megale\inst{1}, Felipe D. Cunha\inst{1} }


\address{Instituto de Ciências Exatas e Informática\\
Pontifícia Universidade Católica de Minas Gerais (PUCMinas)\\
  Belo Horionte -- MG -- Brazil
  \email{fmegale@sga.pucminas.br, felipe@pucminas.br}
}

\begin{document} 

\maketitle

\begin{abstract}
    WIP
\end{abstract}

\begin{resumo}
    WIP
\end{resumo}

\section{Introduction}

With the growth in computer use throughout the decades, the need to communicate with other machines arose, as information needed to travel from one place to another. Connecting these machines into a network was the solution for the information exchange problem. As time passed, communication capabilities became stronger and more popular and computing was not bound to stationary machines anymore. People were now able to send and receive data while on the go with their mobile phones.

Therefore, not only does the network have more nodes connected simultaneously, but also these nodes approach and distance from each other as people go by throughout their daily routines. Furthermore, networking capability became available in vehicles, and this new communication paradigm brought with it some issues, being extensively researched in the literature. We intend to determine if there is a pattern in trips of a vehicle database made available by the City of Austin, Texas, USA, by exploring concepts of complex networks, described in the following paragraphs.

The first concept is assortative mixing, a node characteristic stating whether or not it will connect to another, based on some similarity. The most explored node characteristic to determine network assortativity is the degree of the nodes \cite{Newman-assort-2003}. This characteristic occurs because of homophilic relationships, which are favored in real world when individuals associate with each other based on a common (social or demographic) characteristic \cite{lazarfeld:54}. The criterion we will be using to connect one vehicle to another is 

Since assortative mixing is a characteristic bound to a node, if more nodes behave similarly, then the network presents modularity, or assortative measure for a graph, introduced by \cite{PhysRevE.70.066111}. Modularity states that the nodes in a graph that exhibit homophily have more edges between similar nodes, than between random nodes in an unbiased fashion. The work proposed in \cite{vaanunu:2018} introduce the concept of type assortativity, which states that nodes of different types in a network tend to connect with each other.

Another concept which is being widely studied in the literature is VANETs -- Vehicular Ad-Hoc Networks. Some of the possible uses of this new technology are delivery of traffic information, such as traffic jams, imminent accidents, GPS route recalculation, as well as infotainment for both driver and passengers \cite{abdalla-vanets}. An application of the latter could be the suggestion of a city's attractions to a visiting person or persons, based on the frequency and intensity of the interactions between people who visit such places.

Our object of study are Vehicular Social Networks (VSNs), a social network where the nodes are vehicles. This type of network takes advantage of the traditional VANETs and targets the exploitation of the network's characteristics, such as objectives, interests and mobility patterns. An important field of study in VSNs is content delivery, which is directly connected to the VANETs' architecture \cite{RAHIM201896}.

People often find themselves wondering what are a city's attractions when they are out on vacations, and depending on the place of choice, there is little to no obvious answers for this question. Whether it is due to a shortage on available information, or the visitors simply have not received any recommendations from friends who have been to that destination previously, or they do not know any persons who have also traveled to that destination before.

Of course, some cities have their traditional and most known attractions, such as the Christ the Redeemer status, in Rio de Janeiro, Brazil, or the Statue of Liberty, NY. Other than that, other points that might be of interest for a tourist may remain unknown.

This paper's goal is to exploit a VSN in search of homophily between vehicles in an attempt to deliver content to other vehicles in the network that share some characteristics with those vehicles who already are in a cluster.

\section{Related Works}

The complex network's characteristic chosen to be explored in this paper was defined in \cite{Newman-assort-2003}. In the aforementioned work, Newman defines a measure for assortativity and shows that real social networks often are assortative. Furthermore, \cite{newman_and_park} investigated the correlation of clustering and assortativity in a group. \cite{Newman-assort-2003} and \cite{CATANZARO2004119} show that, degree-wise, the majority of biological and technological networks are disassortative. On the other hand, social networks tend to be assortative (\cite{BUCCAFURRI201556}).

Having defined assortative mixing and homophily, we are able to take advantage of these concepts and explore the possibility of discovering a network, as proposed by \cite{twitter_graph}. This paper studies how the data provided by the users of a social media can be used to determine the existence of a network between them. Still on the matter of targeting content and users, \cite{mulders} present a framework used to exploit social network structures, aiming the improvement of node attribute prediction.

Regarding content distribution in a general context, \cite{Khaitiyakun:2014:ADD:2684793.2684799} proposes a Content Delivery Network (CDN) scheme to transmit data to multiple nodes in the network. This technique involves a node sending information to a replica node, which broadcasts this information to the network. According to this work, the technique proposed was sufficient in a simulated environment to achieve the desired goal.

Still on content distribution, \cite{d2d} propose a framework for delivering information based on D2D communication. The starting point is the assumption that parked vehicles can be used to deliver data to moving vehicles in a VSN. The parked vehicles are used as a maneuver for increasing storage in the network.


% \section{Sections and Paragraphs}

% Section titles must be in boldface, 13pt, flush left. There should be an extra
% 12 pt of space before each title. Section numbering is optional. The first
% paragraph of each section should not be indented, while the first lines of
% subsequent paragraphs should be indented by 1.27 cm.

%\subsection{Subsections}

%The subsection titles must be in boldface, 12pt, flush left.

%\section{Figures and Captions}\label{sec:figs}


%Figure and table captions should be centered if less than one line
%(Figure~\ref{fig:exampleFig1}), otherwise justified and indented by 0.8cm on
%both margins, as shown in Figure~\ref{fig:exampleFig2}. The caption font must
%be Helvetica, 10 point, boldface, with 6 points of space before and after each
%caption.

% \begin{figure}[ht]
% \centering
% \includegraphics[width=.5\textwidth]{fig1.jpg}
% \caption{A typical figure}
% \label{fig:exampleFig1}
% \end{figure}

% \begin{figure}[ht]
% \centering
% \includegraphics[width=.3\textwidth]{fig2.jpg}
% \caption{This figure is an example of a figure caption taking more than one
%   line and justified considering margins mentioned in Section~\ref{sec:figs}.}
% \label{fig:exampleFig2}
% \end{figure}

% In tables, try to avoid the use of colored or shaded backgrounds, and avoid
% thick, doubled, or unnecessary framing lines. When reporting empirical data,
% do not use more decimal digits than warranted by their precision and
% reproducibility. Table caption must be placed before the table (see Table 1)
% and the font used must also be Helvetica, 10 point, boldface, with 6 points of
% space before and after each caption.

% \begin{table}[ht]
% \centering
% \caption{Variables to be considered on the evaluation of interaction
%   techniques}
% \label{tab:exTable1}
% \includegraphics[width=.7\textwidth]{table.jpg}
% \end{table}

% \section{Images}

% All images and illustrations should be in black-and-white, or gray tones,
% excepting for the papers that will be electronically available (on CD-ROMs,
% internet, etc.). The image resolution on paper should be about 600 dpi for
% black-and-white images, and 150-300 dpi for grayscale images.  Do not include
% images with excessive resolution, as they may take hours to print, without any
% visible difference in the result. 

% \section{References}

% Bibliographic references must be unambiguous and uniform.  We recommend giving
% the author names references in brackets, e.g. \cite{knuth:84},
% \cite{boulic:91}, and \cite{smith:99}.

% The references must be listed using 12 point font size, with 6 points of space
% before each reference. The first line of each reference should not be
% indented, while the subsequent should be indented by 0.5 cm.

\bibliographystyle{sbc}
\bibliography{sbc-template}

\end{document}
